\documentclass{article}

\usepackage{graphicx}
\usepackage{lipsum}
\usepackage{subcaption}

\begin{document}

% Setzen Sie ein Bild z.B. von Wikipedia ein und schneiden Sie es mit der Option `trim` zu.
% https://devdocs.io/latex/_005cincludegraphics
% Was passiert, wenn Sie `\caption*{}` verwenden?
\begin{figure}
  \centering
  \includegraphics[width=\linewidth]{example-image-a}
  \caption{Ein Platzhalter mit dem Buchstaben A.}
  \label{fig:img-a}
\end{figure}

Die Abbildung~\ref{fig:img-a} zeigt einen Platzhalter mit dem Buchstaben A.
\lipsum[1]

% Probieren Sie verschiedene Werte für `scale` aus.
% Versuchen Sie, drei Abbildungen mit kleinen Abständen nebeneinander zu platzieren.
\begin{figure}
  \begin{minipage}{\linewidth}
    \begin{minipage}{0.5\linewidth}
      \centering
      \includegraphics[scale=0.4,angle=0]{example-image-a}
      \subcaption{Platzhalter A}
    \end{minipage}% <- sonst wird hier ein Leerzeichen eingefügt
    \begin{minipage}{0.5\linewidth}
      \centering
      \includegraphics[scale=0.4,angle=0]{example-image-b}
      \subcaption{Platzhalter B}
    \end{minipage}
    \caption{Zwei Buchstaben in einer Abbildung.}
  \end{minipage}
\end{figure}

\lipsum[3-5]

\end{document}