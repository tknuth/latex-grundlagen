\documentclass[11pt,a4paper]{article}

% Deaktivieren Sie den Import des Pakets babel und beobachten Sie den Effekt
% auf die Anführungszeichen weiter unten im Text.
\usepackage[ngerman]{babel}

\title{Intro}
\author{Mein Name}

% Was passiert, wenn Sie das Paket babel entfernen?
% Was passiert, wenn Sie dies einkommentieren?
% \date{}

\begin{document}

\maketitle % Ein Titel muss explizit platziert werden

Jeder Absatz beginnt normalerweise mit einer Einrückung.
Diese kann lokal mit \verb|\noindent| oder global
durch Verwendung eines Pakets wie z.B. \verb|parskip| vermieden werden.
Man kann Leerzeichen     beliebig     verwenden,
diese gelten aber nur als ein einzelner freier Raum.
\LaTeX     eignet sich zur Erstellung von wissenschaftlichen Arbeiten.
Zeilenumbrüche müssen mit Leerzeichen oder \verb|\\| erstellt werden,
ansonsten wird der Text zum bestehenden Absatz gezählt.

Man kann in \LaTeX{} Wörter \textbf{fett}, \textit{kursiv} oder \underline{unterstrichen} darstellen.
Der Befehl \verb|\emph{}| hebt kontextsensitiv hervor,
das heißt kursiv im normalen Text und normal im kursiven Text.
Ich kann sogar \textsf{serifenlose Schrift} oder \texttt{Schreibmaschinenschrift} verwenden
und auch \textsc{Kapitälchen} nutzen. Normale "Anführungszeichen" sollten nicht verwendet werden.
Ich kann Text in "`solche"' und ``solche'' Anführungszeichen setzen.\\

\noindent Warum kann man hier nicht einfach \verb|\LaTeX| ohne Zusätze schreiben?\\

\noindent{\LaTeX } eignet sich zur Erstellung von wissenschaftlichen Arbeiten.\\
\noindent\LaTeX{} eignet sich zur Erstellung von wissenschaftlichen Arbeiten.\\
\noindent\LaTeX\ eignet sich zur Erstellung von wissenschaftlichen Arbeiten.\\

Sonderzeichen wie \$, \# und \% lassen sich ebenfalls darstellen,
haben aber in \LaTeX{} eine eigene Bedeutung,
weswegen ihnen ein \textbackslash{} vorangestellt werden muss,
und der \textbackslash{} selbst ist ein \verb|\textbackslash|.

\end{document}
