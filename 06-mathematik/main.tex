\documentclass{article}

\usepackage{graphicx}
\usepackage{amsmath}
\usepackage{lipsum}

\begin{document}

% Üben Sie den Umgang mit Formeln anhand mithilfe von Kapitel 3 aus dem folgenden Buch:
% https://tobi.oetiker.ch/lshort/lshort.pdf

Die Formel \(a^2+b^2=c^2\) ist auch als Satz des Pythagoras bekannt. Die untenstehende Gleichung beschreibt eine lineare Regression. Es gilt entsprechend $\sqrt{a^2+b^2}=c$.

\begin{equation}
  \hat{y} = \hat{\alpha}+\hat{\beta}x + \epsilon
\end{equation}

\begin{equation}
  \mathit{R}^2=\frac{SSR}{SST}=
  \frac{\sum\nolimits \left(\hat{y}_i-\overline{y}\right)^2}{\sum\nolimits\left(y_i-\overline{y}\right)^2}=1-\frac{SSE}{SST}=1-\frac{\sum\nolimits\left(y_i-\hat{y}_i\right)^2}{\sum\nolimits \left(y_i-\overline{y}\right)^2}
\end{equation}

\vspace{1em}
\lipsum[1]

\begin{align}
  \hat{y} & = \hat{\alpha}+\hat{\beta}x \\
  y       & = \alpha+\beta x+\epsilon
\end{align}

\end{document}