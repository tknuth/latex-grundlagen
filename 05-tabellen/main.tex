\documentclass{article}

\usepackage{graphicx}
\usepackage[ngerman]{babel}
\usepackage{booktabs}
\usepackage{multirow}
\usepackage{lipsum}

\begin{document}

% Vergleichen Sie die Positionierung der Tabellen im Code und im PDF-Dokument
% Was fällt Ihnen auf? Lesen Sie in der Dokumentation von `table` und `floats`
% nach, wie Sie die Positionierung der Tabellen beeinflussen können.
% https://devdocs.io/latex/table
% https://devdocs.io/latex/floats

\lipsum[1]

% Erweitern Sie die Tabelle um eine Spalte namens `Quadrat` an
% und schreiben Sie dort die Quadrate der Zahlen hinein
\begin{table}
  \centering
  \begin{tabular}{ccc} % vertikale Linien mit |
    \toprule
    $n$ & Primzahl & Teiler \\
    \midrule
    2   & ja       &        \\
    3   & ja       &        \\
    4   & nein     & 2      \\
    5   & ja       &        \\
    6   & nein     & 2,3    \\
    7   & ja       &        \\
    8   & nein     & 2,4    \\
    \bottomrule
  \end{tabular}
  \caption{Ein paar Zahlen.}
  \label{tab:zahlen1}
\end{table}

\begin{table}
  \centering
  \begin{tabular}{ccc}
    \toprule
    \multirow{2}*{$n$} & \multicolumn{2}{c}{Eigenschaften}          \\
    \cmidrule{2-3}
                       & Primzahl                          & Teiler \\
    \midrule
    2                  & ja                                &        \\
    3                  & ja                                &        \\
    4                  & nein                              & 2      \\
    5                  & ja                                &        \\
    6                  & nein                              & 2,3    \\
    7                  & ja                                &        \\
    8                  & nein                              & 2,4    \\
    \bottomrule
  \end{tabular}
  \caption{Ein paar Zahlen.}
  \label{tab:zahlen2}
\end{table}

\end{document}