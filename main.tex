\documentclass[twocolumn]{article}

\usepackage[ngerman]{babel}
\usepackage{lipsum}
\usepackage[
  style=authoryear,
  autocite=footnote,
  url=false,
  giveninits,
]{biblatex}

% Muss vor \author etc. geladen werden
\usepackage{titling}

\author{Ihr Name}
\date{\today}

\usepackage{fancyhdr}
\pagestyle{fancy}
\fancyhead[L]{\leftmark}
\fancyhead[C]{}
\fancyhead[R]{\theauthor}
\fancyfoot[L,C]{}
\fancyfoot[R]{\thepage}
\renewcommand{\headrulewidth}{0.1pt}
\renewcommand{\footrulewidth}{0.1pt}

\addbibresource{bibliography.bib}

\title{Einführung in \LaTeX}

\begin{document}

\maketitle

\tableofcontents

\section{Erster Abschnitt}

\subsection{Erster Teilabschnitt}

Zum Einstieg in \LaTeX{} existieren hilfreiche Bücher \autocite{oetiker_2022}. Verweise auf Literatur werden mit \verb|\autocite| erzeugt. \citeauthor{mitchell_machine_1997} hat \citeyear{mitchell_machine_1997} ein Buch über maschinelles Lernen mit dem Titel \citetitle{mitchell_machine_1997} geschrieben.

\subsection{Zweiter Teilabschnitt}

\lipsum[1-5]

\subsection{Dritter Teilabschnitt}

\lipsum[6-9]

\printbibliography

\end{document}