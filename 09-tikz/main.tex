\documentclass{article}

\usepackage[ngerman]{babel}
\usepackage{graphicx}
\usepackage{tikz}
\usetikzlibrary{chains}
\usepackage{adjustbox}
\usepackage{kantlipsum}

\begin{document}

\kant[1-3]

\begin{figure}
  \centering
  \begin{tikzpicture}
    % Vervollständigen Sie die Zeichnung zum "Haus vom Nikolaus"
    \tikz \draw[thick] (0,0) -- (1,1) -- (2,0) -- (0,0);
  \end{tikzpicture}
  \caption{Ein unvollständiges "`Haus vom Nikolaus"'.}
\end{figure}

\kant[4-6]

\begin{figure}
  \centering
  \begin{tikzpicture}[level/.style={sibling distance=8em/#1}]
    \node {root}
    child {node {left}
        child {node {leaf}}
        child {node {leaf}}
      }
    child {node {right}
        child {node {leaf}}
        child {node {leaf}}
      };
  \end{tikzpicture}
  \caption{Ein Binärbaum.}
\end{figure}

\kant[7-9]

\begin{figure}
  \centering
  % Was würde ohne \resizebox passieren?
  \resizebox{\textwidth}{!}{
    \begin{tikzpicture}[
        start chain=1 going right,
        node distance=.4cm,
        inner sep=2mm,
        minimum height=1.2cm,
        every node/.style={rounded corners,draw,align=center,join},
        every join/.style={->}
      ]
      \node[on chain=1] {Schritt 1};
      \node[on chain=1] {Schritt 2};
      \node[on chain=1] {Schritt 3};
      \node[on chain=1] {Schritt 4};
      \node[on chain=1] {Schritt 5};
      \node[on chain=1] {Schritt 6};
    \end{tikzpicture}
  }
  \caption{Eine Darstellung eines Prozesses mit mehreren Schritten.}
\end{figure}

\end{document}

